\documentclass[12pt]{article}
\usepackage[utf8]{inputenc}
\usepackage{graphicx}
\usepackage{epstopdf}
\usepackage[spanish]{babel}
%\usepackage[english]{babel}
%\usepackage[latin5]{inputenc}
\usepackage{hyperref}
\usepackage[left=3cm,top=3cm,right=3cm, bottom=3cm,nohead,nofoot]{geometry}
\usepackage{braket}
\usepackage{datenumber}
\usepackage{amsmath,bm}
%\newdate{date}{10}{05}{2013}
%\date{\displaydate{date}}

\begin{document}

\begin{center}
\Huge
Numerical study of the elasticity tensor in the mechanical response of solids to resonant ultrasound modulation

\vspace{3mm}
\Large Jose Alejandro Cubillos Muñoz

\large
201313719


\vspace{2mm}
\Large
Director: Julián Rincón

\normalsize
\vspace{2mm}

\today
\end{center}


\normalsize

\section{Resumen}

Las mediciones ultrasónicas han sido valiosas para el estudio de la materia condensada. Estas permiten determinar las constantes elásticas de los materiales, las cuales son de importancia fundamental, ya que están directamente relacionadas con su estructura atómica. Además, están conectadas a propiedades térmicas de los sólidos mediante la teoría de Debye. Las constantes elásticas, en combinación con mediciones de expansión térmica, también pueden ser usadas para determinar la ecuación de estado de varias funciones termodinámicas \cite{Leisure_1997}.


Mediante la técnica de la espectroscopía de resonancia ultrasónica (RUS por sus siglas en inglés) se puede obtener las mencionadas constantes elásticas. Esta consiste en colocar una muestra del sólido entre dos transductores piezoeléctricos que lo sostienen ligeramente \cite{MIGLIORI19931}. La muestra usualmente tiene formas bien definidas como paralelepípedos rectangulares o esferas y es excitada en un punto por uno de los transductores. Este transductor hace un barrido de frecuencias donde se encuentran varios modos de vibración de la muestra. Mientras el primer transductor hace oscilar la muestra, la respuesta resonante de esta es detectada por el otro transductor. De esta manera, se observa una respuesta amplificada cuando la frecuencia del primer transductor corresponde con la frecuencia de un modo de vibración propio de la muestra, de acuerdo al sistema cristalino de la muestra \cite{Leisure_1997}. Los detalles del montaje experimental se pueden encontrar en la referencia \cite{MIGLIORI19931}.

Aplicando el principio de Hamilton al lagrangiano de un sólido cristalino se obtiene un problema de valores propios generalizado que permite predecir las frecuencias a partir de las constantes elásticas, que se muestra a continuación: \cite{Leisure_1997}:

\begin{equation}
    \rho \omega^2 \bm{E} \vec{a} = \bm{\Gamma} \vec{a}. 
\end{equation}

Los elementos de las matrices $\bm{\Gamma}$ y $\bm{E}$ se obtienen de la siguiente manera \cite{Ohno}: 

\begin{equation}
    \Gamma_{i, \lambda_1, \mu_1,  \nu1; k, \lambda_2, \mu_2, \nu_2} = \sum_{j=1}^{3} \sum_{l=1}^{3} {C_{ijkl} \int_{V}{\frac{\partial \phi_{i \lambda_{1} \mu_1 \nu_1}}{\partial r_j} \frac{\partial \phi_{k \lambda_2 \mu_2 \nu_2}}{\partial r_l} dV}},
\end{equation}

\begin{equation}
    E_{i, \lambda_1, \mu_1,  \nu1; k, \lambda_2, \mu_2, \nu_2} = \int_{V}{\delta_{ik} \phi_{i \lambda_1 \mu_1 \nu_1}  \phi_{k \lambda_2 \mu_2 \nu_2} dV},
\end{equation}

donde $r_1 = x$, $r_2 = y$ y  $r_3 = z$. Las funciones $\phi_{i \lambda \mu \nu}$ son las funciones base en las que se expresan los desplazamientos del sólido al deformarse. Un ejemplo, de lo que pueden ser las funciones base es \cite{Demarest}: 

\begin{equation}
    \phi_{i \lambda \mu \nu} = \left(\frac{x}{L_x} \right)^{\lambda} \left(\frac{y}{L_y} \right)^{\mu} \left(\frac{z}{L_z} \right)^{\nu}.
\end{equation}

Para una geometría de la muestra donde esta es, por ejemplo, un paralelepípedo rectangular y se escogen funciones base con un grado ($N_g$) máximo de 10, se tienen matrices $\bm{\Gamma}$ y $\bm{E}$ de dimensión 858x858. Entre mayor sea el grado de las funciones base, mayor número de frecuencias se podrá predecir. Sin embargo, el tiempo computacional aumentará cúbicamente con el número de grado máximo de las funciones base. A este problema de valores propios se le llama ``forward problem". El problema inverso donde se tienen las frecuencias y se predicen las constantes elásticas se llama el ``inverse problem''.

En el presente proyecto de tesis se creará un modelo de aprendizaje automático que obtenga las constantes elásticas de una muestra dado su espectro de frecuencias. Para ello se resolverán varios ``forward problem'' con distintos valores de constantes elásticas y geometrías. Luego se tomarán los resultados de estos ``forward problem'' para entrenar el modelo que permita resolver el ``inverse problem'' eficientemente.
%Y listo, le enviamos este resumen a Julián 

\section{Abstract}
Ultrasonic measurements have been valuable for the study of condensed matter physics and materials science. Elastic constants, which can be obtained with such measures, are of fundamental importance because they are directly related to the atomic bonding of the material. In addition, they are connected to thermal properties of solids through Debye theory. In combination with specific heat and thermal expansion measurements, elastic constant data can be used to determine the equation of state and various thermodynamic functions \cite{Leisure_1997}.

Elastic constants of a sample can be determined through RUS technique (Resonant Ultrasound Spectroscopy) which consists on placing the sample between two piezoelectric transducers that hold it lightly \cite{MIGLIORI19931}. Usually the sample has a rectangular parallelepiped or sphere shape and is excited at one point by one of the transducers. The frequency of this driving transducer is swept through a range corresponding to a large number of eigenmodes of the sample. The resonant response of the sample is detected by the opposite transducer. A large response is observed when the frequency of the driving transducer corresponds to one of the sample eigenfrequencies \cite{Leisure_1997}. Further details about experimental setup can be found in reference \cite{MIGLIORI19931}.

Applying Hammilton principle on the crystalline solid lagrangian, sample eigenfrequencies can be derived solving a generalized eigenvalue problem, with the elastic constants given \cite{Leisure_1997}:

\begin{equation}
    \rho \omega^2 \bm{E} \vec{a} = \bm{\Gamma} \vec{a}. 
\end{equation}

Matrix elements of $\bm{\Gamma}$ and $\bm{E}$ are \cite{Ohno}: 
 

\begin{equation}
    \Gamma_{i, \lambda_1, \mu_1,  \nu1; k, \lambda_2, \mu_2, \nu_2} = \sum_{j=1}^{3} \sum_{l=1}^{3} {C_{ijkl} \int_{V}{\frac{\partial \phi_{i \lambda_{1} \mu_1 \nu_1}}{\partial r_j} \frac{\partial \phi_{k \lambda_2 \mu_2 \nu_2}}{\partial r_l} dV}},
\end{equation}

\begin{equation}
    E_{i, \lambda_1, \mu_1,  \nu1; k, \lambda_2, \mu_2, \nu_2} = \int_{V}{\delta_{ik} \phi_{i \lambda_1 \mu_1 \nu_1}  \phi_{k \lambda_2 \mu_2 \nu_2} dV},
\end{equation},

where $r_1 = x$, $r_2 = y$ y  $r_3 = z$. Functions $\phi_{i \lambda \mu \nu}$ are the basis functions that describe the displacements of the points in the solid in a deformation. For example, a base function can be \cite{Demarest}: 

\begin{equation}
    \phi_{i \lambda \mu \nu} = \left(\frac{x}{L_x} \right)^{\lambda} \left(\frac{y}{L_y} \right)^{\mu} \left(\frac{z}{L_z} \right)^{\nu}.
\end{equation}

For example, if the sample shape is a parallelepiped, using a ($N_g$) of 10, matrixes $\bm{\Gamma}$ and $\bm{E}$ have dimensions of 858x858. Choosing a greater value of $N_g$ lets the problem to get more eigenfrequencies, but also, computational time will be increased cubicly respect $N_g$. This eigenvalue problem is called the ``forward problem''. Getting the elastic constants using the frequencies is called the ``inverse problem'' 

In the present project a machine learning model will be created to solve the inverse problem. To achieve that, multiple runs of the forward problem will be performed, using diferent elastic constants and shapes to generate data to train the model and solve the inverse problem efficiently. 
%\section{Objetivo General}

%Objetivo general del trabajo. Empieza con un verbo en infinitivo.

%Crear un modelo de Machine Learning que sea capaz de obtener las constantes elásticas de una muestra a partir de su espectro de frecuencias obtenido en RUS



%\begin{thebibliography}{10}
\bibliographystyle{abbrv}
\bibliography{Referencias}

%\end{thebibliography}

\vspace{1.5cm}


\end{document} 