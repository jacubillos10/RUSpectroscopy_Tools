\documentclass[12pt]{article}
\usepackage[utf8]{inputenc}
\usepackage{graphicx}
\usepackage{epstopdf}
\usepackage[spanish]{babel}
%\usepackage[english]{babel}
%\usepackage[latin5]{inputenc}
\usepackage{hyperref}
\usepackage[left=3cm,top=3cm,right=3cm, bottom=2cm,nohead,nofoot]{geometry}
\usepackage{braket}
\usepackage{datenumber}
%\newdate{date}{10}{05}{2013}
%\date{\displaydate{date}}

\begin{document}

\begin{center}
\Huge
Determinación de constantes elásticas de sólidos a partir de las frecuencias de resonancia obtenidas a partir de Espectroscopía de Resonancia Ultrasónica (RUS)

\vspace{3mm}
\Large Jose Alejandro Cubillos Muñoz

\large
201313719


\vspace{2mm}
\Large
Director: Jose Julián Jimenez Rincón

\normalsize
\vspace{2mm}

\today
\end{center}


\normalsize

\section{Resumen}

Las mediciones ultrasónicas de los materiales han sido valiosas para el estudio de la materia condensada. Estas permiten determinar las constantes elásticas de los materiales, las cuales son de importancia fundamental, ya que son segundas derivadas de la energía libre respecto a la deformación y están directamente relacionadas con los enlaces atómicos del material. Además, están conectadas a propiedades térmicas de los sólidos mediante la la teoría de Debye. Los datos de constantes elásticas, en combinación con mediciones de expansión térmica, también pueden ser usados para determinar la ecuación de estado de varias funciones termodinámicas \cite{Leisure_1997}.


Mediante la técnica de la Espectroscopía de Resonancia Ultrasónica (RUS por sus siglas en inglés) se puede obtener las mencionadas constantes elásticas. Esta consiste en colocar una muestra del sólido entre dos transductores piezoeléctricos que lo sostienen ligeramente. La muestra del sólido usualmente tiene formas bien definidas como paralelepípedos o esferas. La muestra es excitada en un punto por uno de los transductores. Este transductor hace un barrido de varias frecuencias donde se encuentras varios modos de vibración de la muestra. Mientras el primer transductor hace oscilar la muestra, la respuesta resonante de esta es detectada por el otro transductor. De esta manera, se observa una respuesta amplia cuando la frecuencia del primer transductor corresponde con la frecuencia de un modo de vibración propio de la muestra \cite{Leisure_1997}. 

%Hablar en el párrafo 3 sobre el problema de valores propios que se genera al plantear el lagrangiano. Forward problem

%Hablar aquí sobre el reverse problem 

%Y listo, le enviamos este resumen a Julián 

\section{Abstract}
Ultrasonic measurements have been valuable for the study of condensed matter physics and materials science. Elastic constants, which can be obtained with such measures, are of fundamental importance; they are second derivatives of the free energy with respect to strain and are directly related to the atomic bonding of the material. In addition, they are connected to thermal properties of solids through Debye theory. In combination with specific heat and thermal expansion measurements, elastic constant data can be used to determine the equation of state and various thermodynamic functions \cite{Leisure_1997}.

Elastic constants of a sample can be determined through RUS technique (Resonant Ultrasound Spectroscopy) which consists on placing the sample between two piezoelectric transducers that hold it lightly. Usually the sample has a a parallelepiped or sphere shape. The sample is excited at one point by one of the transducers. The frequency of this driving transducer is swept through a range corresponding to a large number of eigenmodes of the sample. The resonant response of the sample is detected by the opposite transducer. A large response is observed when the frequency of the driving transducer corresponds to one of the sample eigenfrequencies \cite{Leisure_1997}
\section{Objetivo General}

%Objetivo general del trabajo. Empieza con un verbo en infinitivo.

Crear un modelo de Machine Learning que sea capaz de obtener las constantes elásticas de una muestra a partir de su espectro de frecuencias obtenido en RUS



%\begin{thebibliography}{10}
\bibliographystyle{abbrv}
\bibliography{Referencias}

%\end{thebibliography}

\section*{Firma del Director}
\vspace{1.5cm}


\end{document} 