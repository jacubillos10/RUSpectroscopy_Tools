\documentclass[12pt]{article}
\usepackage[utf8]{inputenc}
\usepackage{graphicx}
\usepackage{epstopdf}
\usepackage[spanish]{babel}
%\usepackage[english]{babel}
%\usepackage[latin5]{inputenc}
\usepackage{hyperref}
\usepackage[left=3cm,top=3cm,right=3cm, bottom=3cm]{geometry}
\usepackage{braket}
\usepackage{datenumber}
\usepackage{amsmath,bm}
\usepackage{breqn}
\usepackage{float}

\usepackage{fancyhdr} % Add this package

%active estos al final
%\pagestyle{fancy} % Use fancy page style
%\fancyhf{} % Clear header and footer
%\fancyfoot[R]{\thepage} % Page number in the bottom left corner
%\renewcommand{\headrulewidth}{0pt} % Remove horizontal line from header
%\renewcommand*{\baselinestretch}{4}
\newcommand*{\Scale}[2][4]{\scalebox{#1}{$#2$}}%
%\newdate{date}{10}{05}{2013}
%\date{\displaydate{date}}
\begin{document}

\begin{center}
\Huge
Estudio numérico del tensor de elasticidad en la respuesta mecánica de sólidos a la modulación de resonancia ultrasónica
%Estudio numérico del tensor de elasticidad en la respuesta de los sólidos a la ¿modulación? ultrasónica resonante

\vspace{3mm}
\Large Jose Alejandro Cubillos Muñoz

\large
201313719


\vspace{2mm}
\Large
Director: Julián Rincón

\normalsize
\vspace{2mm}

\today
\end{center}


\normalsize
\section{Introducción}

Las mediciones ultrasónicas han sido valiosas para el estudio de la materia condensada. Estas permiten determinar las constantes elásticas de los materiales, las cuales son de importancia fundamental, ya que están directamente relacionadas con su estructura atómica. Además, están conectadas a propiedades térmicas de los sólidos mediante la teoría de Debye. Las constantes elásticas, en combinación con mediciones de expansión térmica, también pueden ser usadas para determinar la ecuación de estado de varias funciones termodinámicas \cite{Leisure_1997}.

\subsection {Espectroscopía de resonancia ultrasónica}

Mediante la técnica de la espectroscopía de resonancia ultrasónica (RUS por sus siglas en inglés) se puede obtener las constantes elásticas. Esta consiste en colocar una muestra del sólido entre dos transductores piezoeléctricos que lo sostienen ligeramente \cite{MIGLIORI19931}. La muestra usualmente tiene formas bien definidas, como paralelepípedos rectangulares o esferas, y es excitada en un punto por uno de los transductores. Este transductor hace un barrido de frecuencias donde se encuentran varios modos de vibración de la muestra. Mientras el primer transductor hace oscilar la muestra, la respuesta resonante de esta es detectada por el otro transductor. De esta manera, se observa una respuesta amplificada cuando la frecuencia del primer transductor corresponde con la frecuencia de un modo de vibración propio, de acuerdo al sistema cristalino de la muestra \cite{Leisure_1997}. La figura \ref{fig:montaje_rus} muestra cómo se coloca una muestra paralelepípeda entre los transductores.

\begin{figure}[H]
    \centering
    \includegraphics[scale=0.6]{Img/Montaje_RUS.png}
    \caption{Esquema del montaje experimental para el procedimiento de RUS. Figura tomada de la referencia \cite{Leisure_1997}.}
    \label{fig:montaje_rus}
\end{figure}

Se puede observar que la muestra solo se sostiene de las puntas. Esto permite que hayan condiciones de frontera libres en la muestra. Más detalles del montaje experimental se pueden encontrar en la referencia \cite{MIGLIORI19931}.

\subsection{Los tensores de deformación, esfuerzo y elasticidad}

La respuesta de un material a distintos esfuerzos es gobernada por la ley generalizada de Hooke, la cual está relacionada al tensor de deformación $\bm{\varepsilon}$:

\begin{equation}
    \varepsilon_{ik} = \frac{1}{2} \left(\frac{\partial u_i}{\partial r_k} + \frac{\partial u_k}{\partial r_i} \right).
	\label{eq:deformación}
\end{equation}

El desplazamiento de un punto respecto a su punto de equilibrio es $\vec{u}{(x,y,z)}$. Las componentes de cada vector son: $u_1 = u_x$, $u_2 = u_y$, $u_3 = u_z$, $r_1 = x$, $r_2 = y$ y $r_3 = z$.

Por otro, se define el tensor de esfuerzos, $\bm{\sigma}$, que representa la fuerza por unidad de área que se aplica a un elemento cúbico infinitesimal de la muestra en cada una de las caras, tal como se muestra en la figura \ref{fig:tensor_esfuerzos} \cite{oliveira2020}.

\begin{figure}[H]
    \centering
    \includegraphics[scale=0.3]{Img/Stress_tensor.png}
    \caption{Representación del tensor de esfuerzos en un elemento cúbico infinitesimal de una muestra. Figura tomada de la referencia \cite{oliveira2020}.}
    \label{fig:tensor_esfuerzos}
\end{figure}

Estos dos tensores se relacionan entre sí mediante la ley de Hooke generalizada:
\begin{equation}
	\sigma_{ij} = \sum_{k=1}^{3}\sum_{l=1}^{3}C_{ijkl}\varepsilon_{kl},
\end{equation}
donde $C_{ijkl}$ es una componente del tensor de elasticidad o de constantes elásticas del material $\bm{C}$, el cual se busca obtener en el presente proyecto a partir de las frecuencias de resonancias dadas. 


Se puede observar que el tensor de deformaciones es simétrico; es decir $\varepsilon_{kl} = \varepsilon_{lk}$. Por un lado se tiene que $\sigma_{ij} = C_{ijkl}\varepsilon_{kl}$ y por otro lado se tiene que $\sigma_{ij} = C_{ijlk}\varepsilon_{lk}$. Al igualar las anteriores expresiones se tiene que:
\begin{equation}
	C_{ijkl} = C_{ijlk}.
\end{equation}

Por otra parte, en ausencia de torques se tiene que $\sigma_{ij} = \sigma_{ji}$; es decir, $C_{ijkl}\varepsilon_{kl} = C_{jikl}\varepsilon_{kl}$, lo cual implica que: 
\begin{equation}
	C_{ijkl} = C_{jikl}.
\end{equation}


Las relaciones de simetría encontradas permiten intercambiar los primeros dos índices y los dos últimos indices del tensor de constantes elásticas. Esto reduce la cantidad constantes independientes de 81 a 36. Cada par de índices se puede organizar de 6 formas, es decir, se puede reescribir el tensor de constantes elásticas $\bm{C}$ como una matriz 6$\times$6, cambiando cada par de índices por uno nuevo de acuerdo al cuadro \ref{tab:my-table}. Lo mismo se puede hacer con el tensor de deformación, el cual se puede reescribir como un vector de 6 componentes. Esta forma de representar un tensor simétrico como un vector o bien otro tensor de orden reducido se llama la notación de Voigt \cite{JAMAL2014592}.


\begin{table}[H]
	\centering
	%\begin{tabular}{|c|cccccccccccccccc| }
	\begin{tabular}{|c|c|c|}
	\hline
	índice \textit{i} & índice \textit{j} & índice nuevo \textit{m} \\
	\hline
	1 & 1 & 1            \\
	2 & 2 & 2            \\
	3 & 3 & 3            \\
	2 & 3 & 1            \\
	1 & 3 & 2            \\
	1 & 2 & 3    		 \\	
	\hline       
	\end{tabular}
	\caption{Transformación de pares de índices del tensor de elasticidad y del tensor de deformación.}
	\label{tab:my-table}
\end{table}

Este cambio de índices también permite escribir el tensor de deformaciones como un vector. Con estos nuevos índices se puede escribir la energía potencial elástica, por unidad de volumen, de la siguiente manera: 

\begin{equation}
	\mathcal{U} = \frac{1}{2} \sum_{m=1}^{3}{\sum_{n=1}^{3}{C_{mn}\varepsilon_{m}\varepsilon_{n}}}
	\label{eq:u_vol}
\end{equation}

A partir de aquí se puede ver que los índices \textit{n} y \textit{m} se pueden intercambiar, lo cual implica que $C_{mn} = C_{nm}$. Regresando a la notación con los 4 índices se tiene que $C_{ijkl} = C_{klij}$. Con esto se tiene un tensor $\bm{C}$ con 21 constantes independientes en el caso más general. 


\subsection{Planteamiento del problema de valores propios}

Para determinar las frecuencias de resonancia de una muestra se planteará un lagrangiano que luego será extremizado. Por un lado la energía cinética total de la muestra viene siendo:

\begin{equation}
	K = \frac{1}{2} \int_{V}{\rho \dot{u}^2} dV,
\end{equation}
donde $\rho$ es la densidad de la muestra, $V$ es el volumen y $\dot{u}$ es la velocidad de desplazamiento. Asumiendo un comportamiento periódico en el desplazamiento se puede decir que: $\dot{u} = \omega{u}$ y dado que $u^2 = \sum_{i=1}^{3}{u_i^2} = \sum_{i=1}^{3}{\sum_{k=1}^{3}{\delta_{ik}u_i u_k}}$, la energía cinética queda:

\begin{equation}
	K = \frac{1}{2} \int_{V}{\rho \omega^2 \left(\sum_{i=1}^{3}{\sum_{k=1}^{3}{\delta_{il} u_i u_k}} \right) dV}.
	\label{eq:K}
\end{equation}

Por otro lado, la densidad de energía potencial viene dada por la expresión \ref{eq:u_vol}. Esta se va a reescribir con la notación de 4 índices y se integrará sobre todo el volumen para obtener la energía potencial total: 

\begin{equation}
	U = \frac{1}{2} \int_{V}{\sum_{i=1}^{3}\sum_{j=1}^{3}\sum_{k=1}^{3}\sum_{l=1}^{3}{C_{ijkl}\varepsilon_{ij}\varepsilon_{kl}}dV}
	\label{eq:U_raw}
\end{equation}

Usando la expresión \ref{eq:deformación} en el producto $\varepsilon_{ij}\varepsilon_{kl}$ se tiene que:

\begin{equation}
	\varepsilon_{ij}\varepsilon_{kl} = \frac{1}{4} \left(\frac{\partial u_j}{\partial r_i} \frac{\partial u_k}{\partial r_l} + \frac{\partial u_j}{\partial r_i} \frac{\partial u_l}{\partial r_k}+\frac{\partial u_i}{\partial r_j} \frac{\partial u_k}{\partial r_l}+\frac{\partial u_i}{\partial r_j} \frac{\partial u_l}{\partial r_k} \right).
\end{equation}

Reemplazando el producto $\varepsilon_{ij}\varepsilon_{kl}$ en \ref{eq:U_raw}, separando la ecuación en 4 sumas e intercambiando índices se obtiene la siguiente expresión para la energía potencial:
\begin{equation}
	U = \frac{1}{2} \int_{V}{\sum_{i=1}^{3}\sum_{j=1}^{3}\sum_{k=1}^{3}\sum_{l=1}^{3}{C_{ijkl}\frac{\partial u_i}{\partial r_j} \frac{\partial u_k}{\partial r_l}}dV}.
	\label{eq:U_cooked}
\end{equation}

Usando el método de Rayleigh-Ritz \cite{MIGLIORI19931} se puede expresar las funciones $u_i$ en términos de funciones base $\phi_{i \lambda \mu \nu}$ de la siguiente manera: 
\begin{equation}
	u_i = \sum_{\lambda,\mu,\nu=0}^{\lambda + \mu + \nu \le N_g}{a_{i\lambda\mu\nu} \phi_{i\lambda,\mu,\nu}},
	\label{eq:funciones_base_general}
\end{equation}
donde $N_g$ es el grado máximo de las funciones base, y es un número escogido arbitrariamente. Entre mayor sea este número, mayor número de frecuencias se obtendrá del problema, sin embargo, el tiempo computacional también será mayor, ya que este aumenta en grado 9 a medida que aumenta $N_g$ \cite{Leisure_1997}. Un ejemplo de funciones base puede ser el siguiente \cite{Demarest}:
\begin{equation}
    \phi_{i \lambda \mu \nu} = \left(\frac{x}{L_x} \right)^{\lambda} \left(\frac{y}{L_y} \right)^{\mu} \left(\frac{z}{L_z} \right)^{\nu},
	\label{eq:funciones_base_potencias}
\end{equation}
donde $L_x$, $L_y$ y $L_z$ son las longitudes de la muestra en $x$, $y$ y $z$ respectivamente. Reemplazando la expresión \ref{eq:funciones_base_general} en las ecuaciones \ref{eq:K} y \ref{eq:U_cooked} se obtienen las siguientes expresiones para la energía cinética y potencial respectivamente:
% Aquí falta otra sumatoria con los mu2, nu2...
\begin{equation}
	K = \frac{1}{2} \rho \omega^2 \sum_{i=1}^{3}\sum_{k=1}^{3}\sum_{\Scale[0.5]{\lambda_1,\mu_1,\nu_1 = 0}}^{\Scale[0.5]{\lambda_1 + \mu_1 + \nu_1 \le N_g}} \sum_{\Scale[0.5]{\lambda_2,\mu_2,\nu_2 = 0}}^{\Scale[0.5]{\lambda_2 + \mu_2 + \nu_2 \le N_g}} {a_{i \lambda_1 \mu_1 \nu_1} \left(\int_{V} {\delta_{ik} \phi_{i \lambda_1 \mu_1 \nu_1} \phi_{k \lambda_2 \mu_2 \nu_2} dV} \right) a_{k \lambda_2 \mu_2 \nu_2}},
	\label{eq:K_final}
\end{equation}
\begin{align}
	U = &\;\frac{1}{2} \sum_{i=1}^{3}\sum_{k=1}^{3}\sum_{\Scale[0.5]{\lambda_1,\mu_1,\nu_1 = 0}}^{\Scale[0.5]{\lambda_1 + \mu_1 + \nu_1 \le N_g}} \sum_{\Scale[0.5]{\lambda_2,\mu_2,\nu_2 = 0}}^{\Scale[0.5]{\lambda_2 + \mu_2 + \nu_2 \le N_g}} a_{i \lambda_1 \mu_1 \nu_1} \cdots \nonumber \\ &\left(\sum_{j=1}^{3}\sum_{l=1}^{3} C_{ijkl}\int_{V} { \frac{\partial \phi_{i \lambda_1 \mu_1 \nu_1}}{\partial r_j} \frac{\partial \phi_{k \lambda_2 \mu_2 \nu_2}}{\partial r_l} dV} \right) a_{k \lambda_2 \mu_2 \nu_2}.
	\label{eq:U_final}
\end{align}

Los valores de las constantes $a_{i \lambda \mu \nu}$ se pueden organizar en un vector $\vec{a}$, mientras que los paréntesis en las ecuaciones \ref{eq:K_final} y \ref{eq:U_final} se pueden reorganizar en matrices. De esta manera, podemos expresar la energía cinética y la energía potencial en términos de productos de vectores y matrices: 

\begin{equation}
	K = \frac{1}{2} \rho \omega^2 \vec{a}^{T}\bm{E}\vec{a},
	\label{eq:K_matrix}
\end{equation}

\begin{equation}
	U = \frac{1}{2} \vec{a}^{T}\bm{\Gamma}\vec{a},
\end{equation}
donde los elementos matriciales de $\bm{E}$ y $\bm{\Gamma}$ son los siguientes:

\begin{equation}
	E_{i \lambda_1 \mu_1 \nu_1; k \lambda_2 \mu_2 \nu_2} = \int_{V}{\delta_{ik} \phi_{i \lambda_1 \mu_1 \nu_1}  \phi_{k \lambda_2 \mu_2 \nu_2} dV},
	\label{eq:matriz_E}
\end{equation}

\begin{equation}
    \Gamma_{i, \lambda_1, \mu_1,  \nu1; k, \lambda_2, \mu_2, \nu_2} = \sum_{j=1}^{3} \sum_{l=1}^{3} {C_{ijkl} \int_{V}{\frac{\partial \phi_{i \lambda_{1} \mu_1 \nu_1}}{\partial r_j} \frac{\partial \phi_{k \lambda_2 \mu_2 \nu_2}}{\partial r_l} dV}}.
	\label{eq:matriz_gamma}
\end{equation}

Por ejemplo, usando las funciones base de la ecuación \ref{eq:funciones_base_potencias} y con una muestra de forma paralelepípeda rectangular con dimensiones $L_x$, $L_y$ y $L_z$, se tiene que los elementos de las matrices $\bm{E}$ y $\bm{\Gamma}$, con los valores de $x$, $y$ y $z$ adimensionalizados, son \cite{Ohno}:
\begin{equation}
	E_{i \lambda_1 \mu_1 \nu_1; k \lambda_2 \mu_2 \nu_2} = \int_{-1}^{1} \int_{-1}^{1} \int_{-1}^{1} {\delta_{ik} x^{\lambda_1 + \lambda_2} y^{\mu_1 + \mu_2} z^{\nu_1 + \nu_2} dx dy dz}
	\label{eq:matriz_E_paralelepípedo_pot}
\end{equation}

\begin{equation}
    \Gamma_{i, \lambda_1, \mu_1,  \nu1; k, \lambda_2, \mu_2, \nu_2} = 4\sum_{j=1}^{3} \sum_{l=1}^{3} \frac{C_{ijkl}}{L_j L_l} \int_{-1}^{1} \int_{-1}^{1} \int_{-1}^{1} {\frac{\partial (x^{\lambda_1} y^{\mu_1} z^{\nu_1})}{\partial r_j} \frac{\partial (x^{\lambda_2} y^{\mu_2} z^{\nu_2})}{\partial r_l} dx dy dz}.
	\label{eq:matriz_gamma_paralelepípedo_pot}
\end{equation}


Así, la expresión para el lagrangiano es la siguiente: 

\begin{equation}
	\mathcal{L} = K - U = \frac{1}{2} \rho \omega^2 \vec{a}^{T}\bm{E}\vec{a} - \frac{1}{2} \vec{a}^{T}\bm{\Gamma}\vec{a},
\end{equation}
donde $\vec{a}$ representa a los modos normales de vibración de la muestra. Extremizando este lagrangiano ($\frac{\partial \mathcal{L}}{\partial \vec{a}} = 0$) se obtiene un problema de valores propios generalizado. En estos valores propios se encontrarán las frecuencias de resonancia de la muestra:

\begin{equation}
	\rho \omega^2 \bm{E} \vec{a} = \bm{\Gamma} \vec{a}.
	\label{eq:eigenvalue-problem}
\end{equation}

Mediante la resolución de este problema de valores propios se hallan las frecuencias de resonancia de la muestra teniendo como parámetros sus constantes elásticas, dimensiones y densidad de la muestra. Sin embargo, el objetivo del presente proyecto es hacer el proceso inverso; es decir, crear un método para hallar las constantes elásticas a través de las frecuencias de las muestras obtenidas en el laboratorio. Para ello se probarán distintos modelos de aprendizaje automático, entrenados con datos generados en el problema de valores propios, para resolver el problema de hallar las constantes elásticas. 

\subsection{Estado del arte}

El trabajo más reciente encontrado en donde se usa aprendizaje automático para resolver el problema inverso de valores propios, usa redes neuronales (Deep Learning) para hallar las constantes elásticas. En este trabajo se hace un preprocesamiento del espectro de frecuencias y se transforma en un ``modulated fingerprint''. Este ``modulated fingerprint'' se divide el espectro de frecuencias en ``bins''  que terminan transformados en una imagen. Esta imagen es la que se usa para entrenar la red neuronal \cite{Liu2023}. Se considerará estudiar este método de preprocesamiento de datos para este proyecto, ya que, puede facilitar la fase de entrenamiento de los datos.


Existe otro trabajo donde se hace también un preprocesamiento de los datos, donde se transforma el espectro de frecuencias en una imagen, de tres capas, antes de entrenar a las redes neuronales. En este trabajo en específico se usan dos redes neuronales. Una red saca la información de las constantes en un ``Blackman Diagram'' y la otra red neuronal se encarga de determinar las constantes \cite{Fukuda2023}. 



Por otro lado también se han usado redes neuronales de manera directa para hallar las constantes elásticas de $LiNbO_3$ a la vez que se hallan las constantes de esfuerzo de los piezoeléctricos \cite{Yang2022}. También se ha usado redes neuronales para mapear espectros de frecuencias a dos distintas fases del material de $URu_{2}Si_{2} \cite{Ghosh2020}$. 



Se puede observar que se han hecho trabajos donde se resuelve el problema inverso usando redes neuronales. No se encontraron usos distintos a las redes neuronales para resolver el problema inverso de la espectroscopía de resonancia ultrasónica. Esto da la oportunidad al presente proyecto a probar algoritmos distintos a las redes neuronales, aunque estas se tendrán en cuenta. 

\section{Objetivo General}

Construir modelos de aprendizaje automático consistentes que permitan determinar el tensor de elasticidad de un material, basado en su espectro de resonancia ultrasónica y características geométricas como forma, dimensiones y densidad.
\section{Objetivos Específicos}

%Objetivos espec�ficos del trabajo. Empiezan con un verbo en infinitivo.

\begin{itemize}
	\item Desarrollar software que permita hallar las frecuencias propias de una muestra sólida dada las constantes elásticas, dimensiones y densidad.
	\item Generar grandes conjuntos de datos usando el software, que halla las frecuencias de resonancia, que servirán de entrenamiento al modelo. 
	\item Validar varias frecuencias halladas con datos experimentales de muestras de constantes conocidas.
	\item Entrenar distintos modelos con los datos generados y evaluar su desempeño
\end{itemize}

\section{Metodología}

%Exponer DETALLADAMENTE la metodolog�a que se usar� en la Monograf�a. 

%Monograf�a te�rica o computacional: �C�mo se har�n los c�lculos te�ricos? �C�mo se har�n las simulaciones? �Qu� requerimientos computacionales se necesitan? �Qu� espacios f�sicos o virtuales se van a utilizar?

Para lograr los objetivos planteados se creará un código que resuelva el problema de valores propios para hallar las frecuencias dadas las constantes elásticas. Con este código se busca generar los datos para entrenar el modelo de aprendizaje automático que resolverá el problema inverso. El código construido  no solo generará datos de muestras paralelepípedas sino también muestras cilíndricas y muestras esféricas. Luego se harán mejoras a este código que permitan generar datos más rápido. Una de estas mejoras consiste en reorganizar el vector $\vec{a}$ con las matrices $\bm{E}$ y $\bm{\Gamma}$, de acuerdo a simetrías de reflexión, doble rotación e inversión, para que estas últimas sean diagonales en bloques. Al hacer las matrices diagonales en bloques se reduce drásticamente el tiempo computacional, lo que permite generar datos de manera mucho más rápida \cite{Leisure_1997}. 

Una vez hecho el código que resuelve el problema de valores propios se procederá a calcular las frecuencias de resonancia de varias muestras con dimensiones, constantes elásticas, densidad y espectro de resonancia conocidas. Se compararán los valores propios hallados con el código con las frecuencias experimentales de cada muestra. 

Luego se generarán grandes volúmenes de datos cambiando los parámetros de la forma de la muestra, densidad y constantes elásticas. Para la generación de estos datos se usará el clúster HPC de la Universidad de los Andes. Una vez se ponen a trabajar las máquinas para la generación de datos, se trabajará paralelamente en investigar el funcionamiento de distintos algoritmos de aprendizaje automático y distintos algoritmos de boosting que sean candidatos para la resolución del problema inverso. Algunos algoritmos a los que se investigará su funcionamiento serán: Random Forest, Redes Neuronales y Rotation Forest. Algunos algoritmos de boosting que se investigarán para mejorar el desempeño de los anteriores algoritmos serán adaboost y xgboost. Mientras se hace esta tarea también se investigarán nuevos algoritmos que se vayan conociendo durante el desarrollo del proyecto. 



Una vez se hayan generado los datos se procederá a dividirlos en bloques de entrenamiento, validación y evaluación. Se entrenarán distintos modelos con los datos de entrenamiento y se observará su desempeño con los datos de evaluación. Se escogerá el modelo con el mejor desempeño. También se considerará usar otras técnicas como el k-fold-cross-validation para evitar el ``overfitting'' en donde de dividen los datos en varios bloques, se entrena con los primeros $k-1$ bloques y se evalúa con el último bloque, luego se escoge otro bloque de evaluación y se entrena de nuevo el modelo con los bloques restantes y se vuelve evaluar, repitiendo el proceso k veces.  



\section{Cronograma}

\begin{table}[H]
	\begin{tabular}{|c|cccccccccccccccc| }
	\hline
	Tareas $\backslash$ Semanas & 1 & 2 & 3 & 4 & 5 & 6 & 7 & 8 & 9 & 10 & 11 & 12 & 13 & 14 & 15 & 16  \\
	\hline
	1 & X & X & X & X & X & X & X & X & X & X & X & X & X & X & X & X \\
	2 & X & X & X &   &   &   &   &   &   &   &   &   &   &   &   &   \\
	3 &   &   &   & X & X & X & X & X & X & X & X & X & X &   &   &   \\
	4 &   &   &   & X & X & X & X & X & X & X & X & X & X & X & X & X \\
	5 & X & X &   &   &   &   &   &   &   &   &   &   &   &   &   &   \\
	6 &   &   &   &   &   &   &   &   &   &   &   &   &   & X & X & X \\
	7 &   &   &   & X & X &   &   &   &   &   &   &   &   &   &   &   \\
	\hline
	\end{tabular}
\end{table}
\vspace{1mm}

\begin{itemize}
	\item Tarea 1: Escritura del documento de tesis.
	\item Tarea 2: Creación del código que resuelve el problema de valores propios "forward problem".
	\item Tarea 3: Generación de datos de frecuencias para entrenar los modelos de aprendizaje automático.
	\item Tarea 4: Investigación del funcionamiento de distintos algoritmos de aprendizaje automático y de distintos algoritmos de boosting.
	\item Tarea 5: Realización del curso de cluster HPC en Bloque Neón.
	\item Tarea 6: Entrenamiento de los algoritmos de aprendizaje automático encontrados.
	\item Tarea 7: Comparación de las frecuencias obtenidas con datos experimentales, de muestras cuyas constantes son conocidas. 
\end{itemize} 

\section{Personas Conocedoras del Tema}

%Nombres de por lo menos 3 profesores que conozcan del tema. Uno de ellos debe ser profesor de planta de la Universidad de los Andes.

\begin{itemize}
	\item Paula Giraldo Gallo (Universidad de los Andes)
	\item Edgar Alejandro Marañón León (Universidad de los Andes)
	\item Fabio Arturo Rojas Mora (Universidad de los Andes)
	\item Ferney Rodriguez Dueñas (Universidad de los Andes)
\end{itemize}


%bibliographystyle{abbrv}
\bibliographystyle{ieeetr}
\bibliography{Referencias}

\section*{Firma del Director}
\vspace{1.5cm}

\end{document} 