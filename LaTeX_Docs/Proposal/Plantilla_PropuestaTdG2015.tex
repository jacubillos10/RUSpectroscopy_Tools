\documentclass[12pt]{article}
\usepackage[utf8]{inputenc}
\usepackage{graphicx}
\usepackage{epstopdf}
\usepackage[spanish]{babel}
%\usepackage[english]{babel}
%\usepackage[latin5]{inputenc}
\usepackage{hyperref}
\usepackage[left=3cm,top=3cm,right=3cm, bottom=2cm,nohead,nofoot]{geometry}
\usepackage{braket}
\usepackage{datenumber}
%\newdate{date}{10}{05}{2013}
%\date{\displaydate{date}}

\begin{document}

\begin{center}
\Huge
Determinación de constantes elásticas de sólidos a partir de las frecuencias de resonancia obtenidas a partir de Espectroscopía de Resonancia Ultrasónica

\vspace{3mm}
\Large Jose Alejandro Cubillos Muñoz 2

\large
201313719


\vspace{2mm}
\Large
Director: Jose Julián Jimenez Rincón

\normalsize
\vspace{2mm}

\today
\end{center}


\normalsize
\section{Introducción}

%Introducci�n a la propuesta de Monograf�a. Debe incluir un breve resumen del estado del arte del problema a tratar. Tambi�n deben aparecer citadas todas las referencias de la bibliograf�a (a menos de que se citen m�s adelante, en los objetivos o metodolog�a, por ejemplo)

Aqu\'i texto.


\section{Objetivo General}

%Objetivo general del trabajo. Empieza con un verbo en infinitivo.

Aqu\'i texto.


\section{Objetivos Específicos}

%Objetivos espec�ficos del trabajo. Empiezan con un verbo en infinitivo.

\begin{itemize}
	\item Objetivo 1
	\item Objetivo 2
	\item Objetivo 3
	\item ...
\end{itemize}

\section{Metodología}

%Exponer DETALLADAMENTE la metodolog�a que se usar� en la Monograf�a. 

%Monograf�a te�rica o computacional: �C�mo se har�n los c�lculos te�ricos? �C�mo se har�n las simulaciones? �Qu� requerimientos computacionales se necesitan? �Qu� espacios f�sicos o virtuales se van a utilizar?

%Monograf�a experimental: Recordar que para ser aprobada, los aparatos e insumos experimentales que se usar�n en la Monograf�a deben estar previamente disponibles en la Universidad, o garantizar su disponibilidad para el tiempo en el que se realizar� la misma. �Qu� montajes experimentales se van a usar y que material se requiere? �En qu� espacio f�sico se llevar�n a cabo los experimentos? Si se usan aparatos externos, �qu� permisos se necesitan? Si hay que realizar pagos a terceros, �c�mo se financiar� esto?

Aqu\'i texto.

\section{Cronograma}

\begin{table}[htb]
	\begin{tabular}{|c|cccccccccccccccc| }
	\hline
	Tareas $\backslash$ Semanas & 1 & 2 & 3 & 4 & 5 & 6 & 7 & 8 & 9 & 10 & 11 & 12 & 13 & 14 & 15 & 16  \\
	\hline
	1 & X & X &   &   &   &   &   & X & X &   &   &   &   &   &   &   \\
	2 &   & X & X &   & X & X & X &   &   & X & X & X &   & X & X &   \\
	3 &   &   &   & X &   &   &   & X &   &   &   & X &   &   & X &   \\
	4 & X & X & X & X & X & X & X & X & X & X &   &   &   &   &   &   \\
	5 &   &   &   &   & X &   &   &   & X &   &   & X &   &   & X &   \\
	\hline
	\end{tabular}
\end{table}
\vspace{1mm}

\begin{itemize}
	\item Tarea 1: Descripci\'on de la tarea 1
	\item Tarea 2: Descripci\'on de la tarea 2
	\item Tarea 3: Descripci\'on de la tarea 3
	\item ...
\end{itemize}

\section{Personas Conocedoras del Tema}

%Nombres de por lo menos 3 profesores que conozcan del tema. Uno de ellos debe ser profesor de planta de la Universidad de los Andes.

\begin{itemize}
	\item Nombre de profesor 1 (Instituto o Universidad de afiliaci\'on 1)
	\item Nombre de profesor 2 (Instituto o Universidad de afiliaci\'on 2)
	\item Nombre de profesor 3 (Instituto o Universidad de afiliaci\'on 3)
	\item ...
\end{itemize}


\begin{thebibliography}{10}


\end{thebibliography}

\section*{Firma del Director}
\vspace{1.5cm}

\section*{Firma del Codirector	}



\end{document} 