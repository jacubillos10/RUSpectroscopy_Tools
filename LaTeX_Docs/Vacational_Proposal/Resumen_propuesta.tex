\documentclass[12pt]{article}
\usepackage[utf8]{inputenc}
\usepackage{graphicx}
\usepackage{epstopdf}
\usepackage[spanish]{babel}
%\usepackage[english]{babel}
%\usepackage[latin5]{inputenc}
\usepackage{hyperref}
\usepackage[left=3cm,top=3cm,right=3cm, bottom=3cm,nohead,nofoot]{geometry}
\usepackage{braket}
\usepackage{datenumber}
\usepackage{amsmath,bm}
%\newdate{date}{10}{05}{2013}
%\date{\displaydate{date}}

\begin{document}

\begin{center}
\Huge
Estudio de algoritmos de aprendizaje automático para la resolución del problema inverso al problema de frecuencias propias en la espectroscopía de resonancia ultrasónica. 

\vspace{3mm}
\Large Jose Alejandro Cubillos Muñoz

\large
201313719


\vspace{2mm}
\Large
Director: Julián Rincón

\normalsize
\vspace{2mm}

\today
\end{center}


\normalsize

\section{Objetivos}
\begin{itemize}
    \item Terminar de generar el software que permite resolver el problema de frecuencias propias en la espectroscopía de resonancia ultrasónica (RUS) para geometrías diferentes al paralelepípedo.
    \item Estudiar distintos algoritmos de aprendizaje automático que permitan resolver el problema inverso mencionado en el punto anterior. 
    \item Colaborar en la logística y asistir al evento de HPC summer school. 
\end{itemize}

\section{Plan de Trabajo}
\begin{itemize}
    \item Semanas 1-2: Generación del software del "forward problem" o bien el software que permite hallar el problema de valores propios en el RUS, para distintas geometrías.
    \item Semanas 3-8: Estudio de distintos algoritmos de aprendizaje automático para resolver el problema inverso.
    \item Todas las semanas: Colaboración en la logística del HPC summer school. 
\end{itemize}
%Objetivo general del trabajo. Empieza con un verbo en infinitivo.

%Crear un modelo de Machine Learning que sea capaz de obtener las constantes elásticas de una muestra a partir de su espectro de frecuencias obtenido en RUS



%\begin{thebibliography}{10}
\bibliographystyle{abbrv}
\bibliography{Referencias}

%\end{thebibliography}

\vspace{1.5cm}


\end{document} 