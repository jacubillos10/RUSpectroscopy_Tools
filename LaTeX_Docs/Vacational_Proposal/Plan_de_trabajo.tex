\documentclass[12pt]{article}
\usepackage[utf8]{inputenc}
\usepackage{graphicx}
\usepackage{epstopdf}
\usepackage[spanish]{babel}
%\usepackage[english]{babel}
%\usepackage[latin5]{inputenc}
\usepackage{hyperref}
\usepackage[left=3cm,top=3cm,right=3cm, bottom=3cm,nohead,nofoot]{geometry}
\usepackage{braket}
\usepackage{datenumber}
\usepackage{amsmath,bm}
%\newdate{date}{10}{05}{2013}
%\date{\displaydate{date}}

\begin{document}

\begin{center}
\Huge
Algoritmos de aprendizaje automático para la solución del problema inverso en la espectroscopía de resonancia ultrasónica 

\vspace{3mm}
\Large Jose Alejandro Cubillos Muñoz

\large
201313719


\vspace{2mm}
\Large
Director: Julián Rincón

\normalsize
\vspace{2mm}

\today
\end{center}


\normalsize

\section{Programa de posgrado}
Maestría en Ciencias-Física.
\section{Descripción}

En el periodo vacacional se busca trabajar con el profesor Julián Rincón para continuar con el trabajo de tesis y para colaborar con la logística del evento de 7th HPC Summer School de este año.  La tesis de maestría de Jose Alejandro Cubillos tiene como objetivo lograr resolver el problema inverso de valores propios de la frecuencias en la espectroscopía de resonancia ultrasónica, cuya propuesta fue enviada al comité el 5 de mayo de 2024. En esta propuesta se describe más en detalle en que consiste el problema inverso de la espectroscopía de resonancia ultrasónica, así como su importancia en el estudio de las propiedades de los materiales. A continuación se muestran los objetivos a seguir durante el periodo vacacional para hacer avances en la tesis. 


\subsection{Objetivos}
\begin{itemize}
    \item Desarrollar el software que permite resolver el problema de frecuencias propias en la espectroscopía de resonancia ultrasónica (RUS) para geometrías diferentes al paralelepípedo.
    \item Estudiar distintos algoritmos de aprendizaje automático que permitan resolver el problema inverso mencionado en el punto anterior. 
    \item Colaborar en la logística y asistir al evento de 7th HPC Summer School. 
\end{itemize}

\section{Fecha Inicio}
17 de junio de 2024.
\section{Fecha Final}
31 de julio de 2024.

\section{Nombre del grupo de investigación}
Grupo del Profesor Julián Rincón.

\section{Resultados Esperados}
\begin{itemize}
    \item Terminar el software del problema de las frecuencias propias y comparar los resultados con datos experimentales. 
    \item Construir algoritmos para entrenar al menos un modelo que permita resolver el problema inverso de valores propios. 
    \item Apoyar en la ejecución exitosa del 7th HPC Summer School.
\end{itemize}


\section{Cronograma}
\begin{itemize}
    \item Semanas 1-2: Desarrollo del software del ``forward problem'', o bien, el software que permite hallar el problema de valores propios en el RUS, para distintas geometrías.
    \item Semanas 3-7: Estudio de distintos algoritmos de aprendizaje automático para resolver el problema inverso.
    \item Todas las semanas: Colaboración en la logística del 7th HPC Summer School.
\end{itemize}
%Objetivo general del trabajo. Empieza con un verbo en infinitivo.

%Crear un modelo de Machine Learning que sea capaz de obtener las constantes elásticas de una muestra a partir de su espectro de frecuencias obtenido en RUS



%\begin{thebibliography}{10}
\bibliographystyle{abbrv}
\bibliography{Referencias}

%\end{thebibliography}

\vspace{1.5cm}


\end{document} 